\documentclass[11pt, twoside]{article}

\usepackage{subcaption}
\usepackage{wrapfig}
\usepackage[utf8]{inputenc}
\usepackage[english]{babel}
\usepackage{mathptmx}
\usepackage{graphicx}
\usepackage{fancyhdr}
\usepackage{hyperref}
\usepackage{url}
\usepackage{icomma}
\usepackage{svg}
\graphicspath{{images/}}
\usepackage{footmisc}
\usepackage{listings}
\usepackage{titlesec}
\usepackage{afterpage}
\usepackage[top=2cm,tmargin=2cm,textheight=18cm,footnotesep=1.2cm,bottom=6cm]{geometry}
\usepackage{parskip}
\usepackage{lipsum}
\usepackage{tabularray}
\usepackage{amsmath}
\usepackage{tabu}
%\usepackage{siunitx}
\counterwithin{figure}{section}
\counterwithin{table}{section}
\numberwithin{equation}{section}
\usepackage{setspace}
\usepackage{listings}
\usepackage[labelfont=bf, skip=5pt, font=small]{caption}
\hypersetup{colorlinks=true, linkcolor=blue,filecolor=magenta, urlcolor=cyan, citecolor=blue}
\usepackage[style=apa, sorting=nyt]{biblatex}
\addbibresource{References.bib}
\usepackage[version=4]{mhchem}
\usepackage{pythontex}
\usepackage{enumitem}
\usepackage[dvipsnames, svgnames, usenames]{xcolor}
\usepackage{pdfpages}
\usepackage{cleveref}
\usepackage[T1]{fontenc}
\usepackage{tgpagella}

\lstset{language=Python, 
        basicstyle=\ttfamily\footnotesize, 
        keywordstyle=\color{blue}, 
        commentstyle=\color{green}, 
        breaklines=true,
        stringstyle=\color{red}}

\DeclareCaptionFormat{custom}
{\textbf{#1 #2}\textit{\small #3}}
\captionsetup{format=custom}

\newcommand{\lightgray}[1]{\colorbox{lightgray!40}{\texttt{\detokenize{#1}}}}

% Setlength
\setcounter{tocdepth}{4}
\setcounter{secnumdepth}{4}
\setlength{\parskip}{1em}
\setlength{\parindent}{0em}
\setstretch{0.1}
\setlist{itemsep=0\baselineskip}
\setlength{\skip\footins}{0.5cm}
\setlength{\headheight}{75pt}
\setlength{\belowdisplayskip}{0pt} 
\setlength{\belowdisplayshortskip}{0pt}
\setlength{\abovedisplayskip}{0pt} 
\setlength{\abovedisplayshortskip}{0pt}
\addtolength\abovedisplayskip{-0.5\baselineskip}
\addtolength\belowdisplayskip{-0.5\baselineskip}
\renewcommand{\baselinestretch}{1}\normalsize
%\sisetup{inter-unit-product =$\cdot$}

\newcommand\blankpage{%
    \null
    \thispagestyle{empty}%
    \addtocounter{page}{-1}%
    \newpage}

\crefname{figure}{Fig.}{Figs.}       % Enkelvoud en meervoud voor figuren
\Crefname{figure}{Fig.}{Figs.}       % Hoofdlettervariant
\crefname{table}{Tab.}{Tabs.}        % Enkelvoud en meervoud voor tabellen
\Crefname{table}{Tab.}{Tabs.}        % Hoofdlettervariant
\crefname{equation}{Eq.}{Eqs.}       % Enkelvoud en meervoud voor vergelijkingen
\Crefname{equation}{Eq.}{Eqs.}       % Hoofdlettervariant
\newcommand{\fullref}[1]{\hyperref[#1]{\Cref{#1}}}

\begin{document}

% Eerste pagina zonder fancyhdr en plain maken (normaal)
\thispagestyle{empty}  % Verberg header/footer op de eerste pagina
\pagestyle{fancy}      % Zet fancy header/footer voor de rest van het document

% Verhoog de baselinedichtheid
\renewcommand{\baselinestretch}{1}\normalsize

% Header en footer instellingen
\fancyhf{}  % Leeg de standaard header/footer
\renewcommand{\footrulewidth}{1.1pt}  % Voeg een voetregel toe
\renewcommand{\headrulewidth}{0pt}    % Geen regel in de header
\fancyhead[L]{\includegraphics[width=0.3\textwidth]{HAN-merkteken-descriptor.png}}  % Linksboven
\fancyfoot[LE]{\thepage}  % Links onder pagina nummer (even pagina's)
\fancyfoot[RO]{\thepage}  % Rechts onder pagina nummer (oneven pagina's)
\fancyfoot[LO]{\leftmark}   % Voeg hoofdstuktitel toe in de voettekst
\fancyfoot[RE]{\leftmark}   % Voeg hoofdstuktitel toe in de voettekst

% Begin van het document
\begin{titlepage}
    \centering
    \includegraphics[width=0.7\textwidth]{HAN-merkteken-descriptor.png}
    \vspace{1cm} % Voeg ruimte toe
    \hrule
    \vspace{0.5cm}
    {\Huge \bfseries Portfolio Assignment 1 - Ethics \par}
    \vspace{0.5cm}
    \hrule
    \vspace{0.5cm}
    {\LARGE Marcello Wienhoven \par}
    \vspace{0.5cm}
    {\LARGE \today \par}
    \vspace{0.5cm}
    \hrule
    
\end{titlepage}

\pagenumbering{roman} 

\newpage
\pagenumbering{arabic}
\setcounter{page}{1}

\section{Two Ethical Models}
The two ethical models I have chosen to discuss are the Care Ethics model and the Pragmatic Ethics model. These models are further explained in subsection \ref{sub:care-ethics-model} and \ref{sub:pragmatic-ethics-model} respectively.

\subsection{Care Ethics Model}\label{sub:care-ethics-model}
The Care Ethics model emerged in the 1980's from feminist ethics (\cite{uvh-care-ethics}). The model seeks to maintain relationships by contextualizing and promoting the well-being of care-givers and care-receivers in a network of relationships (\cite{iep-care-ethics}). It emphasizes empathy, compassion, and the interconnectedness of individuals within a community. In practice, this means that when faced with an ethical dilemma, one should consider how their actions will affect those they care about and strive to maintain and nurture those relationships.

I have chosen this model because I believe that in today's world, where technology and data science are becoming increasingly prevalent, it is essential to remember the human aspect of our work. Data scientists often work with sensitive information and make decisions that can significantly impact individuals and communities. By adopting a care ethics approach, data scientists can ensure that they prioritize the well-being of those affected by their work, fostering trust and promoting ethical practices in the field.

\subsection{Pragmatic Ethics Model}\label{sub:pragmatic-ethics-model}
The Pragmatic Ethics model is one of the three main schools under the umbrella of ethical relativism (\cite{gotquestions-pragmatic-ethics}). This model suggests that ethical decisions should be based on practical consequences and the context of the situation rather than adhering to absolute moral principles. This general idea has attracted a remarkably rich and at times contrary range of interpretations, including (\cite{sep-pragmatism}): 

\begin{itemize}
    \item[-] that all philosophical concepts should be tested via scientific experimentation, that a claim is true if and only if it is useful;
    \item[-] that articulate language rests on a deep bed of shared human practices that can never be fully ‘made explicit’.
\end{itemize}

I have chosen this model because I believe that in the field of data science, it is crucial to consider the practical implications of our work. Data scientists often face complex ethical dilemmas that require balancing competing interests and values. By adopting a pragmatic ethics approach, data scientists can make decisions that are contextually appropriate and consider the potential consequences of their actions on various stakeholders.

\clearpage
\section{Similarities, Differences, and Preferences}
In subsection \ref{sub:similarities-and-differences}, I will discuss the similarities and differences between the Care Ethics and Pragmatic Ethics models. In subsection \ref{sub:preferences}, I will explain which model I prefer and why.

\subsection{Similarities and Differences}\label{sub:similarities-and-differences}
Both the Care Ethics and Pragmatic Ethics models emphasize the importance of context in ethical decision-making. They both recognize that ethical dilemmas are often complex and require a nuanced approach that considers the specific circumstances and relationships involved.

However, the two models differ in their primary focus. The Care Ethics model centers on relationships and the well-being of individuals within a community, while the Pragmatic Ethics model focuses on practical consequences and the context of the situation. The Care Ethics model is more relational and empathetic, whereas the Pragmatic Ethics model is more outcome-oriented and flexible.

\subsection{Preferences}\label{sub:preferences}
I prefer the Pragmatic Ethics model because it allows for a more flexible and contextually appropriate approach to ethical decision-making. In the field of data science, where situations can vary widely and involve multiple stakeholders with competing interests, a pragmatic approach enables data scientists to consider the practical implications of their actions and make decisions that are best suited to the specific circumstances. Additionally, the Pragmatic Ethics model encourages critical thinking and adaptability, which are essential skills for data scientists navigating the rapidly evolving landscape of technology and data.

The Care Ethics model is undoubtedly valuable, especially in fostering empathy and maintaining relationships. However, I believe that the Pragmatic Ethics model provides a more comprehensive framework for addressing the complex ethical challenges faced by data scientists today.

I also think that the Pragmatic Ethics model is best suited for the field of data science because it encourages a balance between ethical principles and practical considerations. Data scientists often need to make decisions that involve trade-offs between competing values, such as privacy, transparency, and innovation. The Pragmatic Ethics model allows for a more nuanced approach that considers the specific context and potential consequences of these decisions, ultimately leading to more ethical and effective outcomes.

\clearpage

% Bibliography
\printbibliography[heading=bibintoc]

\end{document}
