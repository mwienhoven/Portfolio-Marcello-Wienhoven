\documentclass[11pt, twoside]{article}

\usepackage{subcaption}
\usepackage{wrapfig}
\usepackage[utf8]{inputenc}
\usepackage[english]{babel}
\usepackage{mathptmx}
\usepackage{graphicx}
\usepackage{fancyhdr}
\usepackage{hyperref}
\usepackage{url}
\usepackage{icomma}
\usepackage{svg}
\graphicspath{{images/}}
\usepackage{footmisc}
\usepackage{listings}
\usepackage{titlesec}
\usepackage{afterpage}
\usepackage[top=2cm,tmargin=2cm,textheight=18cm,footnotesep=1.2cm,bottom=6cm]{geometry}
\usepackage{parskip}
\usepackage{lipsum}
\usepackage{tabularray}
\usepackage{amsmath}
\usepackage{tabu}
%\usepackage{siunitx}
\counterwithin{figure}{section}
\counterwithin{table}{section}
\numberwithin{equation}{section}
\usepackage{setspace}
\usepackage{listings}
\usepackage[labelfont=bf, skip=5pt, font=small]{caption}
\hypersetup{colorlinks=true, linkcolor=blue,filecolor=magenta, urlcolor=cyan, citecolor=blue}
\usepackage[style=apa, sorting=nyt]{biblatex}
\addbibresource{References.bib}
\usepackage[version=4]{mhchem}
\usepackage{pythontex}
\usepackage{enumitem}
\usepackage[dvipsnames, svgnames, usenames]{xcolor}
\usepackage{pdfpages}
\usepackage{cleveref}
\usepackage[T1]{fontenc}
\usepackage{tgpagella}

\lstset{language=Python, 
        basicstyle=\ttfamily\footnotesize, 
        keywordstyle=\color{blue}, 
        commentstyle=\color{green}, 
        breaklines=true,
        stringstyle=\color{red}}

\DeclareCaptionFormat{custom}
{\textbf{#1 #2}\textit{\small #3}}
\captionsetup{format=custom}

\newcommand{\lightgray}[1]{\colorbox{lightgray!40}{\texttt{\detokenize{#1}}}}

% Setlength
\setcounter{tocdepth}{4}
\setcounter{secnumdepth}{4}
\setlength{\parskip}{1em}
\setlength{\parindent}{0em}
\setstretch{0.1}
\setlist{itemsep=0\baselineskip}
\setlength{\skip\footins}{0.5cm}
\setlength{\headheight}{75pt}
\setlength{\belowdisplayskip}{0pt} 
\setlength{\belowdisplayshortskip}{0pt}
\setlength{\abovedisplayskip}{0pt} 
\setlength{\abovedisplayshortskip}{0pt}
\addtolength\abovedisplayskip{-0.5\baselineskip}
\addtolength\belowdisplayskip{-0.5\baselineskip}
\renewcommand{\baselinestretch}{1}\normalsize
%\sisetup{inter-unit-product =$\cdot$}

\newcommand\blankpage{%
    \null
    \thispagestyle{empty}%
    \addtocounter{page}{-1}%
    \newpage}

\crefname{figure}{Fig.}{Figs.}       % Enkelvoud en meervoud voor figuren
\Crefname{figure}{Fig.}{Figs.}       % Hoofdlettervariant
\crefname{table}{Tab.}{Tabs.}        % Enkelvoud en meervoud voor tabellen
\Crefname{table}{Tab.}{Tabs.}        % Hoofdlettervariant
\crefname{equation}{Eq.}{Eqs.}       % Enkelvoud en meervoud voor vergelijkingen
\Crefname{equation}{Eq.}{Eqs.}       % Hoofdlettervariant
\newcommand{\fullref}[1]{\hyperref[#1]{\Cref{#1}}}

\begin{document}

% Eerste pagina zonder fancyhdr en plain maken (normaal)
\thispagestyle{empty}  % Verberg header/footer op de eerste pagina
\pagestyle{fancy}      % Zet fancy header/footer voor de rest van het document

% Verhoog de baselinedichtheid
\renewcommand{\baselinestretch}{1}\normalsize

% Header en footer instellingen
\fancyhf{}  % Leeg de standaard header/footer
\renewcommand{\footrulewidth}{1.1pt}  % Voeg een voetregel toe
\renewcommand{\headrulewidth}{0pt}    % Geen regel in de header
\fancyhead[L]{\includegraphics[width=0.3\textwidth]{HAN-merkteken-descriptor.png}}  % Linksboven
\fancyfoot[LE]{\thepage}  % Links onder pagina nummer (even pagina's)
\fancyfoot[RO]{\thepage}  % Rechts onder pagina nummer (oneven pagina's)
\fancyfoot[LO]{\leftmark}   % Voeg hoofdstuktitel toe in de voettekst
\fancyfoot[RE]{\leftmark}   % Voeg hoofdstuktitel toe in de voettekst

% Begin van het document
\begin{titlepage}
    \centering
    \includegraphics[width=0.7\textwidth]{HAN-merkteken-descriptor.png}
    \vspace{1cm} % Voeg ruimte toe
    \hrule
    \vspace{0.5cm}
    {\Huge \bfseries Portfolio Assignment 2 - Ethics \par}
    \vspace{0.5cm}
    \hrule
    \vspace{0.5cm}
    {\LARGE Marcello Wienhoven \par}
    \vspace{0.5cm}
    {\LARGE \today \par}
    \vspace{0.5cm}
    \hrule
    
\end{titlepage}

\pagenumbering{roman} 

\newpage
\pagenumbering{arabic}
\setcounter{page}{1}

\section{First Impression Dating Casus}

\section{Directed Acyclic Graph (DAG)}

\section{Second Impression Dating Casus}

\section{Recommendations for a Data Scientist in This Case}

\clearpage

% Bibliography
\printbibliography[heading=bibintoc]

\end{document}
